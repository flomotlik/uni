\documentclass{article}

\begin{document}

%Title Page
\author{Florian Motlik}
\title{Scripting}
\date{\today}
\maketitle

%TOC
\tableofcontents
\newpage

\begin{abstract}
%TODO
Some Abstract Text
%TODO give description of word Script
\end{abstract}

\section{What is a Scripting Language}
This Section describes what a Scripting language is, it's characteristics and
the reasons for using a scripting language in your project.

Scripting languages can be traditionally described as lightweight easy to use
languages for small scripting and embedded purposes. For example scripting
languages like PERL or Python are heavily used in server infrastructure to
provide common tasks. Their easy, but powerful syntax often leads to an increased
productivity in writing applications. For those reasons they also often used to
script bigger applications like games.

Although they started as languages for small tasks scripting languages have
already outgrown that field. They are currently used for big desktop as well
as web development. Examples would be Linux desktop applications written in
Python or web development with Ruby on Rails.
\subsection{Characteristics} \label{sec-characteristics}
This section will give an overview over the characeteristics of scripting
languages.

A first important distinction regarding scripting languages is between
data-definition and runtime scripting languages. Data-definition languages are
used to specify in-game objects often in a declarative way. Runtime languages,
the main topic of this paper, are used to control events occuring inside the
game and engine.

\begin{description}
\item[execution mode] Scripting languages are generally interpreted by a
virtual machine (e.g. Ruby) or JIT-compiled(e.g. Python). JIT-compilation
compiles the source or an intermediate format (e.g. Java byte code) upon first
execution. This allows for optimizations for specific platforms.
%TODO find Resource for JIT compilation
\item[lightweightness] As scripting languages need to either run in embedded
systems or need best performance they are generally very lightweight.
%TODO noch bisserl überarbeiten
\item[rapid iteration] On the fly reloading and interpretation are two methods
for achieving rapid iteration. Generally game engines, as most bigger software
products, need to support rapid iteration for embedded scripting languages to
rapid development. Through their dynamicity Scripting languages also support
more rapid iteration and more powerful runtime features.
\item[domain specifity] Scripting languages are often created to suite a
specific need. For example resource handling, error handling or cuncurrency can
be done in a way to prevent errors from happening. Errors can be prevented
through syntax tailored to a special need.
\end{description}

\subsection{Why use a Scripting language}
This section reasons why to use a scripting language in a broader sense than
for game development alone, but uses game development as an example.

Scripting languages are currently gaining lot of ground in Software Engineering.
% TODO find resource
Through their dynamic nature they allow, as explained in
\ref{sec-characteristics}, a more rapid iteration and more powerful runtime
features. This, in combination with TDD\footnote{Test Driven Development}, can
lead to high quality software with quicker development. Especially in
Web-Development 

\section{Scripting languages in the wild}
Which scripting languages are there and how are they used.

\section{Scripting and the Game Engine}
How does a scripting language interact with the Game Engine environment.

\end{document}
