\documentclass{article}

\begin{document}

%Title Page
\author{Florian Motlik - 0425525}
\title{Scripting in Game Engines}
\date{\today}
\maketitle

%TOC
\tableofcontents
\newpage

\begin{abstract}
This paper discusses the Scripting chapter of \cite{gregory2009game}. Basics
of a scripting language as well as interaction with the Game engine will presented.

Scripting languages are used today to rapidly iterate in the development of software.
They often provide an easy and domain specific syntax. Together with game engine 
characteristics like event handling or multithreading they provide a fast and scalable
approach for developing games.
\end{abstract}

\section{What is a Scripting Language}
This Section describes what a Scripting language is, it's characteristics and
the reasons for using a scripting language in your project.

Scripting allows to react to or control events and processes inside a 
Game Engine in a lightweigth and easily changeable fashion.
%http://www.cameronalbert.com/post/2008/08/Scripting-for-Games.aspx

Scripting languages can be traditionally described as lightweight easy to use
languages for small scripting and embedded purposes. For example scripting
languages like PERL or Python are heavily used in server infrastructure to
provide common tasks. Their easy, but powerful syntax often leads to an increased
productivity in writing applications. For those reasons they are also often used to
script bigger applications like games.

Although they started as languages for small tasks scripting languages have
already outgrown that field. They are currently used for big desktop as well
as web development. Examples would be Linux desktop applications written in
Python or web development with Ruby on Rails.
\subsection{Characteristics} \label{sec-characteristics}
This section will give an overview over the characeteristics of scripting
languages.

A first important distinction regarding scripting languages is between
data-definition and runtime scripting languages. Data-definition languages are
used to specify in-game objects often in a declarative way. Runtime languages,
the main topic of this paper, are used to control events occuring inside the
game and engine.

\begin{description}
\item[execution mode] Scripting languages are generally interpreted by a
virtual machine (e.g. Ruby) or JIT-compiled(e.g. Python). JIT-compilation
compiles the source or an intermediate format (e.g. Java byte code) upon first
execution. This allows for optimizations for specific platforms.
\item[lightweightness] As scripting languages need to either run in embedded
systems or need best performance they are generally very lightweight.
\item[rapid iteration] On the fly reloading and interpretation are two methods
for achieving rapid iteration. Generally game engines, as most bigger software
products, need to support rapid iteration for embedded scripting languages. 
Through their dynamicity Scripting languages also support
more rapid iteration and more powerful runtime features.
\item[domain specifity] Scripting languages are often created to suite a
specific need. For example resource handling, error handling or cuncurrency can
be done in a way to prevent errors from happening. Errors can be prevented
through syntax tailored to a special need.
\end{description}

\subsection{Why use a Scripting language}
This section reasons why to use a scripting language in a broader sense than
for game development alone, but uses game development as an example.

Scripting languages are currently gaining lot of ground in Software Engineering.
Through their dynamic nature they allow, as explained in
\ref{sec-characteristics}, a more rapid iteration and more powerful runtime
features. This, in combination with TDD\footnote{Test Driven Development}, can
lead to high quality software with quicker development. Especially in
Web-Development, where fast iteration cycles are predominant nowadays,
scripting languages have lots of potential.

As Game Development, even more than Web Development, relies on trial and error
fast iteration cycles are a must. Consider the development of an FPS game. Very
often these sort of games take place in a very linear fashion and are heavily
scripted. Certain actions in a game, like getting into an area or using some
object, trigger an event which then executes a script. Trying multiple
variations of a trigger as well as a script are necessary to find the right
balance in the game. In a non domain specific language lots of overhead would
be needed to satisfy the constraints of the language. In a domain specific
scripting language on the other hand syntax optimisations could already be
built in.

As scripting languages often support a wider or easier accessible range of
meta programming functions building domain specific languages with them can be
much easier.

To summarize scripting languages can support faster iterations and
therefore more trial and error during a project.

\section{Scripting and the Game Engine}
Interaction between scripts and the game engine will be the topic of this
section.
\subsection{Use-Cases in Game Engines}
Scripting can be used in several scenarios inside a game engine.

\begin{description}
\item[Callbacks] Scripted callbacks can be used to extend the functionality in
small parts. The game is largely implemented in the native language of the
engine, but for some events scripts get called. The scripts are not used for
any specific purpose, but are probably scattered throughout the system. This
leads to the game being easier to change by users or other developers.
\item[Event Handlers] Scripts that get specifically called to react to events
happening in the game are Event Handlers. The event handling process will be
described in more detail in subsection \ref{event-handling}
\item[Script driven engine] Some engines, like Panda 3D, use a script language
as their primary programming language to develop the game. Native libraries are
only used for speed critical features and are called from the scripts. All non
critical code is written in the scripting language and thus easier to change.

\end{description}

\subsection{Native language Interface}
As scripting languages tend to be much slower than their native counterparts
some features need to be implemented in a lower level language. Consider for
example a hardware intensive and often used algorithm like matrix transformations. They
need to be run often and take considerable time. Thus a low level library written
for the specific environment (OS, Processor) can give a huge performance benefit.

Native language interfaces have to be supported by the scripting runtime. It can
take names and values for a specific call and map those onto the native language,
which will probably be C/C++. Different languages implement this feature either
transparently, like Python, or more directly. Python uses method tables to store
every method an object has. These can be implemented in C without the user of the
library seeing any difference to a default python implementation.

\subsection{Game Object Handles}
As scripts have to interact with different game objects the way to address them
becomes very important. As game objects can potentially be implemented in a
native language a mapping from the native languages way to address an object
(e.g. pointers) to the scripts way (e.g. name) must be implemented. Three
different ways to address that problem are
\begin{itemize}
\item numeric ids
\item string ids
\item hashed string ids
\end{itemize}

Integer ids have the advantage of fast runtime performance, but the disadvantage
that they are hard to read and interpret by humans. String ids on the other hand
are easy to interpret by humans, but show much slower runtime performance. A 
compromise between the two is hashing the strings to integers during compilation
or when they are run first. This way they are still readable during development,
but have much better runtime performance.

\subsection{Event Handling} \label{event-handling}
Events are used ubiquitously in game engine systems. Events are mostly sent to and
processed by objects. These objects can have scripts attached that react to a
certain event being fired. Thus scripts can be reused for different scenarios and
objects. Objects can also have several scripts added for a specific event. Thus
the whole system becomes very loosely coupled and thus easy to change.

As well as objects scripts can be attached to other entities in a game. Regions
or specific time intervals are two possibilities.

Another very powerful feature is if scripts not only handle events, but also
send them. This way the coupling between different scripts can be loosened
even more.

\subsection{Multi-Threading}
With todays multi core infrastructure multi-threading becomes prevailant in lots of
scenarios. Game engines are no different in that regard.

The main difference in a game engine multithreading environment to other environments
is that scheduling is done via cooperative multitasking. With this scheduling
technique the script is not interrupted, but yields control over the environment
at specific predefined states. This way no action or animation in the game is suddenly
interrupted.
When yielding the control scripts can define certain events to wait for, so they
get reactivated again. The game engine can then suspend the script and wake it up
accordingly.


\bibliographystyle{plain}
\bibliography{references}
\end{document}
